\section{Conclusion} 
%Although 

The MPI shared memory programming capabilities introduced by the MPI-3 standard was evaluated. Some comparison to other shared memory programming model (OpenMP) were performed. It was found that, in general, the performance of well tuned $MPI_{sm}$ programs is similar to their OpenMP equivalent.



\subsection*{Pure Shared Memory Programing Model}

Although performance of $MPI_{sm}$ programs can be as good as their OpenMP equivalent, its the lack of support of  work-sharing scheduling in loops work-sharing constructs (dynamic, guided, etc.) require more effort on the part of the programmer. That could harm its acceptance. Additionally, being a relatively new shared memory programming model, make it harder to compete with well established 20+ years old OpenMP.


\subsection*{Hybrid Programing Model}

As a hybrid programing model $MPI_{sm}$ could have some advantages over OpenMP. Among them, it is worth to mention:

\begin{itemize} 
  \item The programmer do not have to worry about the level of thread support.
  \item Application libraries do not need to be thread-safe.
  \item Although still can exist, less prone to race conditions.
  \item No interference between the affinity policies of MPI and OpenMP, which can harm performance.  
  \item No thread overhead.
\end{itemize}



